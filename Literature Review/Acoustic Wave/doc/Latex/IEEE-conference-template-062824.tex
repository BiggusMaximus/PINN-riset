\documentclass[conference]{IEEEtran}
\usepackage{booktabs, tabularx}
\usepackage[referable]{threeparttablex}
\usepackage{stfloats}
\IEEEoverridecommandlockouts
% The preceding line is only needed to identify funding in the first footnote. If that is unneeded, please comment it out.
%Template version as of 6/27/2024

\usepackage{cite}
\usepackage{amsmath,amssymb,amsfonts}
\usepackage{algorithmic}
\usepackage{graphicx}
\usepackage{textcomp}
\usepackage{xcolor}
\def\BibTeX{{\rm B\kern-.05em{\sc i\kern-.025em b}\kern-.08em
    T\kern-.1667em\lower.7ex\hbox{E}\kern-.125emX}}
\begin{document}

\title{Conference Paper Title*\\
% {\footnotesize \textsuperscript{*}Note: Sub-titles are not captured for https://ieeexplore.ieee.org  and
% should not be used}
}

\author{\IEEEauthorblockN{Annastya Bagas Dewantara}
\IEEEauthorblockA{\textit{Departement of Electrical Engineering} \\
\textit{Universitas Indonesia}\\
Depok, Indonesia \\
\textbf{NPM: 2306185574} \\
annastya.bagas@ui.ac.id}
}

\maketitle

\begin{abstract}
This document is a model and instructions for \LaTeX.
This and the IEEEtran.cls file define the components of your paper [title, text, heads, etc.]. *CRITICAL: Do Not Use Symbols, Special Characters, Footnotes, 
or Math in Paper Title or Abstract.
\end{abstract}

\begin{IEEEkeywords}
component, formatting, style, styling, insert.
\end{IEEEkeywords}

\section{Introduction}
Wireless Sensor Network (WSN) telah menjadi salah satu teknologi penting yang telah digunakan masa modern ini karena kemampuannya dalam melakukan monitoring dan melakukan pengumpulan data dari berbagai bidang-bidang berbeda, seperti kesehatan [1], militer [2], agrikultur [3], industri [4], dan pencegahan bencana alam [5]. Network ini terdiri atas sensor-sensor otonom yang tersebar secara spatial untuk melakukan pengumpulan data dan melakukan pengiriman ke base station. Sensor tersebut harus memiliki efisiensi tinggi, konsumsi daya yang rendah serta jangkauan komunikasi yang jauh. Tidak hanya itu, sensor harus memiliki akurasi yang tinggi dalam melakukan pembacaan parameter serta reabilitas dalam melakukan pengiriman data secara real-time. 

Salah satu faktor yang mempengaruhi tingkat reabilitas, skalabilitas dan efisiensi dari WSN adalah protokol routing yang digunakan. Kamal dan Al-Karaki membagi traditional network structure berdasarkan flat-based, hierarhical-based, dan location-based yang dapat terbagi kembali berdasarkan protocol routing seperti multi-path-based, query-based, negotion-based, quality-of-service (QoS) dan coherent-based [6]. Pada hierarhical-based routing seperti Low Energy Adaptive Cluster Hierarchy (LEACH) memiliki kekurangan bahwa pemilihan cluster head (CH) dan pembentukan cluster dipilih secara acak, flat-based routing protocol seperti Sensor Protocols for Information via Negotiation (SPIN) memiliki permasalahan pada skalabilitas, overhead dan redundancy, sedangkan pada location-based routing protocol seperti Greedy Perimeter Stateless Routing (GSPR) memiliki permasalahan pada void handling dan konsumsi energy yang disebabkan akibat greedy forwarding.   Masing-masing permasalahan tersebut dapat disebabkan akibat peraturan terhadap routing table maupun jalur pengiriman data bersifat statis sehingga WSN tidak adaptif dalam menghadapi dinamika topologi WSN.

Meningkatnya perkembangan artificial intelligence (AI) telah merevolusi cara manusia dalam melakukan pengambilan keputusan. Algoritma AI dibagi atas tiga kategori yakni; supervised, unsupervised serta reinforcement learning. Algoritman reinforcement learning (RL) merupakan salah satu teknik AI yang melakukan pembelajaran melalui interaksi terhadap lingkungannya. Algoritma RL telah banyak digunakan di berbagai bidang atas kemampuannya dalam melakukan pengenalan pola berdasarkan informasi data yang diberikan. Salah satu implementasi dari algoritma RL tersebut adalah pada peningkatan efisiensi energi melalui optimasi protokol routing WSN [7], [8], [9], [10], [11]. Algoritma RL bekerja menggunakan informasi jaringan sebagai input masukan dalam melakukan pengambilan keputusan dari protokol routing yang digunakan. Informasi jaringan tersebut dapat berupa energy node, pola mobilitas, densitas traffic, topologi network, channel atau jangkauan pembacaan. Sehingga, memungkinkan WSN dalam melakukan perubahan secara real-time dan adaptif dalam menghadapi perubahan jaringan yang dinamis sebagai upaya dalam meningkatkan lifespan maupun throughput dari WSN.

Meskipun pertumbuhan dan perkembangan aplikasi dari algoritma RL pada WSN terus meningkat, akan tetapi review komprehensive pada topik ini masih belum banyak dilakukan. Komprehensif review pada penerapan algoritma RL pada protokol routing yang tersedia masih terfokus pada titik penerapan tertentu seperti pada flying adhoc network [12], underwater wireless sensor network (UWSN) [13] dan distributed wireless network [14]. Literatur review ini bertujuan untuk menyediakan scoping review dalam menyediakan pemetaan sistematis dan sintesis dari literature yang ada terkait penggunaan algoritma reinforcement learning pada WSN. Melalui hasil dari sintesis dan pemetaan tersebut, review ini bertujuan untuk melakukan identifikasi terkait trend, gaps dan tantangan sebagai arah dan panduan dalam melakukan penelitian kedepannya. Dalam mencapai hal tersebut penelitian ini menggunakan systematic mapping untuk mengkategorikan dan memvisualisasikan distribusi penelitian yang ada serta pendekatan yang digunakan berdasarkan parameter yang ditentukan. The Prefered Reporting Items for Systematic Review and Meta-Analyses (PRISMA) framework digunakan sebagai panduan dalam melakukan identifikasi, screening serta inclusion dan ekslusi dari penelitian relevan. Tidak hanya itu, teknik natural language processing (NLP) juga digunakan dalam mengekstrak teks data serta informasi untuk mendapatkan insight serta informasi baru dari literatur yang dikumpulkan dengan jumlah besar. Tujuan dari literatur review ini adalah untuk menghimpun artikel-artikel untuk mengetahui tren, gap, serta keterbaharuan dari penelitian yang dilakukan yang menunjuang penelitian disertasi yang berfokus pada pengembangan Deep reinforcement learning pada protokol routing untuk meningkatkan efisinesi energi pada WSN.


\section{State of the Art}
Before you begin to format your paper, first write and save the content as a 

\begin{table*}[t]
    \centering
    \begin{threeparttable}
        \caption{My three-part table}
        \label{tab:widetable}
        \begin{tabular}{cr rr p{6cm} rr rr rr}
            \toprule
            \multicolumn{1}{c}{\textbf{Lorem}} & 
            \multicolumn{1}{c}{\textbf{a}} & 
            \multicolumn{1}{c}{\textbf{b}} & 
            \multicolumn{1}{c}{\textbf{c}} & 
            \multicolumn{1}{c}{\textbf{d}} & 
            \multicolumn{1}{c}{\textbf{e}} & \\
            \midrule
            lorem & xxxx & xxxx & xxxx & 
            roses pencarian serta scanning literature dilakukan secara otonom menggunakan teknik NLP untuk memungkinkan pengumpulan data dalam jumlah besar melalui input parameter sebagai search terms untuk mengidentifikasi artikel yang memiliki relevansi. Proses pencarian menggunakan beberapa properti seperti keyword, properties, property synonyms, property groups, start and end year [15]. Literature dikumpulkan dari beberapa sumber seperti IEEE, ACM, Elsevier, Springer, and Wiley. Dalam mendeterminasi initial pool dari artikel yang nantinya dievaluasi untuk elegibilitas, pencarian dilakukan menggunakan beberapa search query yakni (“reinforcement learning” OR “RL”) AND (“protocol routing” OR “routing strategy” OR “routing algorithm”) AND (“Wireless Sensor Network” OR “WSN” OR “Sensor Network”) AND “energy”. Tidak hanya itu, proses pencarian juga dilakukan menggunakan properties yang ditunjukkan pada Table 1. & xxxx & xxxx \\
            lorem & xxxx & xxxx & xxxx & xxxx & xxxx & xxxx \\
            \bottomrule
        \end{tabular}
        \begin{tablenotes}[para] \footnotesize
            \note{lorem ipsum}
        \end{tablenotes}
    \end{threeparttable}
\end{table*}



\section{Methodology}
Proses pencarian serta scanning literature dilakukan secara otonom menggunakan teknik NLP untuk memungkinkan pengumpulan data dalam jumlah besar melalui input parameter sebagai search terms untuk mengidentifikasi artikel yang memiliki relevansi. Proses pencarian menggunakan beberapa properti tersebut seperti keyword, properties, property synonyms, property groups, start and end year [15]. Literature dikumpulkan dari beberapa sumber seperti IEEE, ACM, Elsevier, Springer, and Wiley. Dalam mendeterminasi initial pool dari artikel yang nantinya dievaluasi untuk elegibilitas, pencarian dilakukan menggunakan beberapa search query yakni (“reinforcement learning” OR “RL”) AND (“protocol routing” OR “routing strategy” OR “routing algorithm”) AND (“Wireless Sensor Network” OR “WSN” OR “Sensor Network”) AND “energy”. Tidak hanya itu, proses pencarian juga dilakukan menggunakan properties yang ditunjukkan pada Table. 1 yang mengandung sinonim atau akronim dari search query untuk menghindari adanya ekslusi dini pada proses pencarian. Properties tersebut dikategorikan berdasarkan suatu group secara thematically dan sematically pada Table. 1 NLP Toolkit digunakan dalam mencari artikel berdasarkan boolean condition dari search query berdasarkan informasi title dan abstrak yang di dapatkan dari masing-masing artikel. Artikel yang memenuhi boolean condition tersebut kemudian di lakukan pengecekan oleh human reader untuk mengetahui konteks serta relevansi dari artikel.

\subsection{Research Questions}
Research question utama yang diajukan dari review ini ditunjukkan sebagai berikut :
RQ1: What are the prominent research topics and publication trends in literature?	
RQ2: In which way reinforcement learning can improve efficiency energy using routing protocol in WSN?
RQ3: What are future research directions for reinforcement learning algorithm in protocol routing for WSN?

\subsection{Inclussion and Exclussion Criteria}
Pada fase screening artikel yang telah di seleksi manual maupun diseleksi melalui NLP toolkit, di eliminasi menggunakan inclussion dan exclussion criteria yang ditetapkan. Inklusi serta ekslusi yang ditetapkan sebagai berikut:
IC1	Artikel tersebut ditulis dalam bahasa inggris dan terpublished pada rentang tahun antara Januari 2019 hingga Agustus 2024.
IC2	Artikel tersebut setidak-tidaknya mengandung 3 huruf dari propoerties atau search query pada abstrak maupun judulnya.
IC3	Artikel yang dikumpulkan merupakan artikel yang telah melalui peer-reviewed serta bukan secondary manuscript seperti survey, systematic review maupun mapping, atau jenis reveiw lainnya

\section{Methodology}
Dalam proses pencarian literatur, total 3.770 artikel diidentifikasi dari beberapa basis data, termasuk Springer (n=1116), Elsevier (n=564), IEEE (n=282), ACM (n=1334), dan Wiley (n=474). Setelah menghapus 939 artikel yang terduplikasi, sebanyak 2.831 artikel disaring lebih lanjut. Dari hasil penyaringan ini, 2.226 artikel dinilai kelayakannya untuk masuk dalam tinjauan. Namun, 2.033 artikel dikeluarkan karena tidak memenuhi kriteria yang ditentukan. Pada akhirnya, sebanyak 193 studi dipilih untuk dimasukkan dalam tinjauan akhir.
Dari total 3.770 artikel yang diidentifikasi, proses seleksi awal menghapus 939 artikel yang merupakan duplikasi, sehingga menyisakan 2.831 artikel untuk disaring. Penyaringan ini dilakukan untuk menilai relevansi setiap artikel terhadap topik penelitian, terutama dalam hal penggunaan algoritma pembelajaran penguatan (reinforcement learning) dalam protokol routing hierarkis seperti LEACH. Dari artikel yang disaring, sebanyak 2.226 artikel dipertimbangkan untuk kelayakan lebih lanjut berdasarkan kriteria spesifik, seperti relevansi dengan protokol LEACH dan penerapan algoritma DQN untuk routing jaringan. Namun, setelah dilakukan evaluasi mendalam terhadap metodologi, hasil, dan kontribusi ilmiah, sebanyak 2.033 artikel dikeluarkan karena tidak memenuhi standar atau relevansi yang diperlukan untuk kajian ini. Pada akhirnya, 193 studi dipilih sebagai bahan utama untuk tinjauan literatur ini, yang mencakup penelitian tentang algoritma pembelajaran penguatan, optimasi energi dalam WSN, dan penerapan protokol routing hierarkis seperti LEACH yang ditampilkan dalam taksonomi yang diilustrasikan pada Fig. 2 dan Fig. 3.
Pengambilan artikel dari masing-masing publisher dilakukan dengan menggunakan teknik web scraping yang terstruktur dan sistematis. Proses ini melibatkan pengumpulan data dari berbagai basis data seperti Springer, Elsevier, IEEE, ACM, dan Wiley. Setiap artikel yang sesuai dengan search query yang diinputkan diambil secara otomatis dengan teknik scraping, di mana informasi penting dari setiap artikel diekstraksi secara langsung dari halaman web masing-masing publisher. Setiap elemen ini diperoleh dengan cara mengambil tampilan atau struktur HTML dari masing-masing halaman artikel yang dicari melalui search query. Pendekatan ini memungkinkan pengumpulan data yang lebih cepat dan akurat dari basis data yang berbeda, sekaligus mempermudah analisis dan penyaringan artikel yang relevan dengan penelitian. Teknik web scraping yang digunakan memanfaatkan berbagai pustaka Python seperti BeautifulSoup dan Selenium untuk mempermudah navigasi dan ekstraksi data dari situs web, di mana struktur HTML dari setiap halaman diproses secara otomatis untuk menghasilkan dataset yang kaya dan siap digunakan untuk analisis literatur lebih lanjut yang digerakkan secara massive dan otomatis menggunakan bot seperti yang ditunjukkan pada Fig. 4.


\bibliography{references}

\vspace{12pt}


\end{document}
